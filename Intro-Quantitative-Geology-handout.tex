\documentclass[11pt,a4paper]{article}

\renewcommand{\rmdefault}{ptm}
\renewcommand{\sfdefault}{ptm}

\usepackage[colorlinks=true,urlcolor=blue]{hyperref}

\usepackage[T1]{fontenc}
\usepackage[utf8]{inputenc}
\usepackage[margin=2cm]{geometry}

\usepackage{fancyhdr}
\pagestyle{fancy}
\headheight = 15pt
\lhead{}
\chead{}
\rhead{}
\lfoot{}
\cfoot{\thepage}
\rfoot{}

\begin{document}

\title{Introduction to Quantitative Geology (Course 54070)\\Spring 2016}
\author{}
\date{}
\maketitle

\section*{Course meetings}
Mondays 14-16, C108, Physicum (14.3-2.5)\\
Wednesdays 14-16 \textit{and} 16-18, D211, Physicum (16.3-4.5)\\
\textit{Work sessions on Mondays 16-18, A113, Physicum (11.4-2.5)}

\section*{Instructor}
David Whipp\\
Office: D430, Exactum\\
Email: \href{mailto:david.whipp@helsinki.fi}{\texttt{david.whipp@helsinki.fi}}\\
Phone: (0)2 941 51617

\section*{Course assistant}
Jorina Sch\"{u}tt\\
Office: D422, Exactum\\
Email: \href{mailto:jorina.schutt@helsinki.fi}{\texttt{jorina.schutt@helsinki.fi}}\\
Phone: (0)45 1865288

\section*{Course website}
Main course site: \href{https://github.com/Intro-Quantitative-Geology}{\texttt{https://github.com/Intro-Quantitative-Geology}}\\
Moodle page: \href{https://moodle.helsinki.fi/course/view.php?id=12453}{\texttt{https://moodle.helsinki.fi/course/view.php?id=12453}}

\section*{Textbooks}
There are \textbf{no required textbooks} for this course. This course uses a wide range of sources for course information and the main textbooks are given below.\\


\noindent Recommended textbooks (in order of relevance):
\begin{itemize}
  \item Zelle, J. (2010) \textit{Python Programming: An Introduction to Computer Science}, Second edition. Franklin, Beedle \& Associates.
  \item St\"{u}we, K. (2007) \textit{Geodynamics of the Lithosphere}, Second edition. Springer.
  \item Braun, J., van der Beek, P. and Batt, G. (2006) \textit{Quantitative Thermochronology: Numerical Methods for the Interpretation of Thermochronological Data}, First edition. Cambridge University Press.
  \item Taylor, J. R. (1997) \textit{An Introduction to Error Analysis: The Study of Uncertainties in Physical Measurements}, Second edition. University Science Books.
\end{itemize}
Optional textbooks:
\begin{itemize}
  \item Pelletier, J. (2008) \textit{Quantitative Modeling of Earth Surface Processes}, First edition. Springer.  
  \item Trauth, M. H. (2010) \textit{MATLAB\textsuperscript{\textregistered} Recipes for Earth Sciences}, Third edition. Springer.
  \item Beazley, D. M. (2012) \textit{Python Essential Reference}, Fourth edition. Addison-Wesley.
\end{itemize}


\section*{General description of the course}
The course aims to:
\begin{itemize}
  \item Introduce students to modeling Earth science data and the Python programming language
  \item Develop basic programming skills through analysis of common equations used in the Earth sciences
  \item Present common techniques for comparing geologic data to numerical model predictions
\end{itemize}

\section*{Course format}
This course is equally divided between lectures on Mondays and computer-based laboratory exercises on Wednesdays.
Monday classes will be divided into two $\sim$45 minute lectures with a short break in the middle of class.
Lectures slides will be available on Moodle on the morning prior to each lecture.
Laboratory exercises will focus on applying lecture material and developing basic programming skills using the Python language.
Typical exercises will involve a short introduction, followed by topical computer-based tasks.
At the end of the exercises, you will be asked to submit answers to relevant questions, and possibly related plots and/or Python codes you have written/used.
Students are encouraged to discuss and work together on the laboratory exercises, however the laboratory summary write-ups that you submit must be completed individually and must clearly reflect your own work.

\section*{Grading}
Course grades will be given using the universal six-level grading scale from 0 to 5.
Your grade will be based on (1) seven laboratory exercise summary write-ups, and (2) a course project (briefly described below).
The weight of each item is given below.
\begin{itemize}
  \item 50\% - Exercise write-ups (7 in total)
  \item 50\% - Final project report
\end{itemize}
Note: Deadlines for exercise write-ups and the term project are \textbf{firm} and given in the schedule on the following pages.
Exercise write-ups will be due by the start of lab on the due date.
If you anticipate you will not be able to submit any of these items by the given deadline, you should let me know as early as possible and \textbf{must} let me know at least one day in advance.
Late write-ups will be marked down 25\% per day late, so please submit it on time.

\section*{Final project}
The final project is based on the results you will produce in Exercises 6 and 7.
In these exercises, you will modify a Python code to read in a data file, make some basic calculations using some of the equations we've discussed earlier in the course and produce a series of plots.
The goal of the exercise is to model a geological dataset and use the model to interpret the data.
The final project report will involve writing a short paper with the introductory and background material for the data from Exercises 6 and 7, presenting your results in a series of plots with a short section of text and then discussing the meaning of the results.
The intent is for you to write a short scientific paper with the same material that would typically be present in a scientific journal article.
Details on the final project paper will be given later in the course.

\newpage

\section*{Course topics by week}
\textit{Lecture content, readings and due dates are subject to change.}

\subsection*{Basic concepts in quantitative geology}
\textbf{14.3 (Lecture)} - (I) What is Quantitative Geology?, (II) Essentials of computing
\begin{itemize}
  \item Readings: St\"{u}we, Chapter 1; Zelle, Chapters 1 \& 2
\end{itemize}
\textbf{16.3 (Lab)} - Exercise 1: Introduction to Python and NumPy I
\begin{itemize}
  \item Homework: None
\end{itemize}

\subsection*{Dealing with age data: Radioactivity and essential geostatistics}
\textbf{21.3 (Lecture)} - (I) Common statistical methods in geoscience, (II) What do geochronologic ages mean?
\begin{itemize}
  \item Readings: Taylor, Chapters 2 \& 4
\end{itemize}
\textbf{23.3 (Lab)} - Exercise 2: Introduction to Python and NumPy II
\begin{itemize}
  \item Homework: Exercise 1 write-up due
\end{itemize}

\subsection*{28.3, 30.3 - NO CLASS (Easter holiday break)}

\subsection*{Hillslope sediment transport and heat transfer: The diffusion equation}
\textbf{4.4 (Lecture)} - (I) Natural diffusion: Hillslope sediment transport, Earth's thermal field, (II) Applying the diffusion equation
\begin{itemize}
  \item Readings: St\"{u}we, Chapter 3; Pelletier, Chapter 2
\end{itemize}
\textbf{6.4 (Lab)} - Exercise 3: Uplift and diffusion of the Earth's surface
\begin{itemize}
  \item Homework: Exercise 2 write-up due
\end{itemize}

\subsection*{Fluvial incision and rock uplift: The advection/wave equation}
\textbf{11.4 (Lecture)} - (I) Advection of the Earth's surface: Fluvial incision and rock uplift, (II) Applying the advection/wave equation
\begin{itemize}
  \item Readings: St\"{u}we, Chapter 3; Pelletier, Chapter 4
\end{itemize}
\textbf{13.4 (Lab)} - Exercise 4: River profile calculations
\begin{itemize}
  \item Homework: Exercise 3 write-up due
\end{itemize}

\subsection*{Viscous flow of rock and ice: (Non-)Newtonian flow equations}
\textbf{18.4 (Lecture)} - (I) Rocks and ice as viscous materials, (II) Equations of viscous flow
\begin{itemize}
  \item Readings: St\"{u}we, Chapter 5; Pelletier, Chapter 6
\end{itemize}
\textbf{20.4 (Lab)} - Exercise 5: Glacier mechanics
\begin{itemize}
  \item Homework: Exercise 4 write-up due
\end{itemize}

\subsection*{Quantitative thermochronology: Linking ages to processes}
\textbf{25.4 (Lecture)} - (I) Basic concepts in thermochronology, (II) Low-temperature thermochronology
\begin{itemize}
  \item Readings: Braun et al., Chapters 1-3
\end{itemize}
\textbf{27.4 (Lab)} - Exercise 6: Predicting thermochronometer ages I
\begin{itemize}
  \item Homework: Exercise 5 write-up due
\end{itemize}

\noindent\textbf{2.5 (Lecture)} - (I) Quantifying erosion with thermochronology, (II) Thermochronology for landscape evolution
\begin{itemize}
  \item Readings: Braun et al., Chapters 1-3
\end{itemize}
\textbf{4.5 (Lab)} - Exercise 7: Predicting thermochronometer ages II
\begin{itemize}
  \item Homework: Exercise 6 write-up due
\end{itemize}

\subsection*{Final project deadline}
\textbf{13.5} - Final project (includes Exercise 7) due by 17.00

\end{document}
